
\section{CONCLUSION}

In this project, we have considered how SiPMs can potentially be used as part of a spark chamber trigger system in replacement of traditional PMTs. A model of the SiPM has been built using software that simulates electrical circuits. These simulations allow us to simulate the functionality of a SiPM, providing further insight on how SiPMs operate and how individual parameters affect its performance. These simulations have been shown to be consistent with theory.

However, a SiPM model relies heavily on knowing the parameters for a specific SiPM. Due to circumstances, we have not yet been able to determine the parameters of the specific SiPM we have in the laboratory. These parameters can be determined through the methods discussed in this report.

Thus, the next steps are clear. We would need to be able to test the SiPM experimentally to further assess its suitability in a trigger system. Features such as the DCR can then be measured quantitatively to see the SiPM's real-world performance. Having a working model can be complementary to such a task since real-world results can be fed back into the model.

Finally, there are also a lot of potential further simulations that can be run using this model. For example, additional cells can be added in parallel to the existing model to simulate optical crosstalk. Then, there are methods to simulate the circuit such that the rate of optical crosstalk is in line with experimental measurements.
