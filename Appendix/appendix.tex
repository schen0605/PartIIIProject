
\onecolumn
\LARGE
\noindent\textbf{APPENDICES}
\normalsize
\appendix

\section{EFFECT OF COVID-19}

The COVID-19 pandemic has meant that Part III students were unable to carry out any experimental work at the Cavendish Labratory. As a result, this project has been adapted to be one of a computational nature. Initially, the plan for this project was to be able to handle and perform experiments on the SiPMs directly in order to assess their suitability for operating in a spark chamber. However, due to the pandemic, we relied on the simulation of a SiPM using equivalent circuit models.

\section{SPICE OUTPUT FILE}

\begin{verbatim}
  time	V(sipm_out)
  Step Information: C_q=12.2f  (Run: 1/3)
  0.000000000000000e+00	4.748553e-04
  1.000000003627494e-15	4.748553e-04
  2.000000007254987e-15	4.748553e-04
  4.000000014509975e-15	4.748553e-04
  8.000000029019950e-15	4.748553e-04
  ...
  4.996054687500080e-08	5.146755e-04
  5.000000000000000e-08	5.145579e-04
  Step Information: C_q=22.2f  (Run: 2/3)
  0.000000000000000e+00	4.748553e-04
  1.000000003627494e-15	4.748553e-04
  4.000000014509975e-15	4.748553e-04
  ...
  Step Information: C_q=32.2f  (Run: 3/3)
  0.000000000000000e+00	4.748553e-04
  1.000000003627494e-15	4.748553e-04
  2.000000007254987e-15	4.748553e-04
  ...
  4.996054687500080e-08	5.244602e-04
  5.000000000000000e-08	5.243570e-04
\end{verbatim}

\section{EXAMPLE CODE}

Some examples of code used in the project are presented here. An example of a SPICE output file is presented in appendix B and the subsequent code is used to treat such files. In general, the code parses the SPICE output file and separates the different variable values into its own subarray. We can then perform calculations on the arrays and plot any necessary graphs.

\subsection{example1.py}

\begin{lstlisting}[language=Python, basicstyle=\footnotesize, numbers=left, numberstyle=\tiny, breaklines=true, commentstyle=\itshape\color{magenta}, keywordstyle=\bfseries\color{blue}, identifierstyle=\color{cyan}, stringstyle=\color{red}, showstringspaces=false]

# ----------------------------
# AUTHOR: [REDACTED]
# crsID: [REDACTED]
# COLLEGE: [REDACTED]
# DATE: Feb 2021
# PURPOSE: Part III Project - Creates a graph using matplotlib from a LTSpice output data file (see appendix B).
# ----------------------------

import numpy as np
import matplotlib.pyplot as plt

counter = 0 # used to separate multiple data streams e.g. an IV graph with different resistance values

# ----------------------------
# SPECIFY FILE HERE:
# ----------------------------

f = open("FILE_PATH","r")

# ----------------------------
# Initialises global data arrays
# ----------------------------

# Ignore header lines
f.readline()
f.readline()

# iterates through every line in the Spice output file and determines the number of data streams
for line in f:
    line = line.strip()
    row = line.split()
    try: # test if the line contains data values
        floatTest = float(row[1])
    except ValueError:
        counter += 1

# very generic variable names but avoids confusion when parsing different files with different variables e.g. resistance/capacitance data
xVar = [[] for i in range(counter+1)]
yVar = [[] for i in range(counter+1)]
graphLabel = np.array([])

# ----------------------------
# Helper functions
# ----------------------------

def appendArray(data, count):
    global xVar, yVar
    xVar[count] = np.append(xVar[count],float(data[0]))
    yVar[count] = np.append(yVar[count],float(data[1]))
    return

# ----------------------------
# Main function
# ----------------------------

# reset back to beginning of the file
counter = 0
f.seek(0)
f.readline()

# iterates through every line in the Spice output file again and adds every line of data into an array
for line in f:
    # print(repr(line))
    line = line.strip()
    row = line.split()
    # variable[0] = np.append(variable[0],1.1)
    # print(variable)

    try: # test if the line contains data values
        floatTest = float(row[0])
    except ValueError:
        counter += 1
        graphLabel = np.append(graphLabel,row[2])
        continue # moves on to next line if the line does not contain data

    appendArray(row, counter - 1)

f.close()

# ----------------------------
# Plot graphs
# ----------------------------

fig = plt.figure(figsize=(12, 7))
for i in range(len(xVar)):
    label = graphLabel[i][4:-1] # extracts the parameter value
    plt.plot(xVar[i]*1E9, yVar[i]*1E3, linewidth = 2, label = label)

plt.xlabel('Time [ns]')
plt.ylabel('Voltage [mV]')
plt.legend(frameon=False)

plt.savefig("FILE_PATH")
plt.show()

\end{lstlisting}

\subsection{example2.py}

\begin{lstlisting}[language=Python, basicstyle=\footnotesize, numbers=left, numberstyle=\tiny, breaklines=true, commentstyle=\itshape\color{magenta}, keywordstyle=\bfseries\color{blue}, identifierstyle=\color{cyan}, stringstyle=\color{red}, showstringspaces=false]

# ----------------------------
# AUTHOR: [REDACTED]
# crsID: [REDACTED]
# COLLEGE: [REDACTED]
# DATE: Feb 2021
# PURPOSE: Part III Project - Parses through a specified number of files and performs a linear regression of the data in each file.
# ----------------------------

import numpy as np
import matplotlib.pyplot as plt
from scipy import optimize

counter = 0 # used to separate multiple data streams e.g. an IV graph with different resistance values
numFiles = 4 # number of files to loop over

# ----------------------------
# SPECIFY FILE HERE:
# ----------------------------

f = open("FILE_PATH","r")

# ----------------------------
# Initialises global data arrays
# ----------------------------

# Ignore header lines
f.readline()
f.readline()

for line in f:
    line = line.strip()
    row = line.split()
    try: # test if the line contains data values
        floatTest = float(row[1])
    except ValueError:
        counter += 1

voltageRs1 = [[] for i in range(counter+1)] # Rs = 1
voltageRs2 = [[] for i in range(counter+1)] # Rs = 10
voltageRs3 = [[] for i in range(counter+1)] # Rs = 50
voltageRs4 = [[] for i in range(counter+1)] # Rs = 100

f.close()

# ----------------------------
# Helper functions
# ----------------------------

def appendArray(data, count, array):
    array[count] = np.append(array[count],float(data[1]))
    return

def linFunc(x, m):
    return m*x

# ----------------------------
# Main function
# ----------------------------

# Loops over number of files
for i in range(1, numFiles + 1):
    counter = 0
    filePath = "FILE_PATH" + str(i) + ".txt" # opens text files named xxx1.txt, xxx2.txt, ...
    f = open(filePath, "r")

    f.readline()

    for line in f:
        line = line.strip()
        row = line.split()

        try: # test if the line contains data values
            floatTest = float(row[0])
        except ValueError:
            counter += 1
            continue # moves on to next line if the line does not contain data

        if(i == 1):
            appendArray(row, counter - 1, voltageRs1)
        elif(i == 2):
            appendArray(row, counter - 1, voltageRs2)
        elif(i == 3):
            appendArray(row, counter - 1, voltageRs3)
        else:
            appendArray(row, counter - 1, voltageRs4)

    f.close()

# find the peak value of a single fired cell and calculates ratio: peak / single cell peak
firedCells = np.array(range(1, len(voltageRs1) + 1))
singleCellPeak1 = np.amax(voltageRs1[0])
singleCellPeak2 = np.amax(voltageRs2[0])
singleCellPeak3 = np.amax(voltageRs3[0])
singleCellPeak4 = np.amax(voltageRs4[0])

for i in range(len(voltageRs1)):
    voltageRs1[i] = np.amax(voltageRs1[i]) / singleCellPeak1
    voltageRs2[i] = np.amax(voltageRs2[i]) / singleCellPeak2
    voltageRs3[i] = np.amax(voltageRs3[i]) / singleCellPeak3
    voltageRs4[i] = np.amax(voltageRs4[i]) / singleCellPeak4

# ----------------------------
# Linear fit
# ----------------------------

params, covariance = optimize.curve_fit(linFunc, firedCells, voltageRs1)
params2, covariance2 = optimize.curve_fit(linFunc, firedCells, voltageRs2)
params3, covariance3 = optimize.curve_fit(linFunc, firedCells, voltageRs3)
params4, covariance4 = optimize.curve_fit(linFunc, firedCells, voltageRs4)

print params, params2, params3, params4

\end{lstlisting}

\subsection{example3.py}

\begin{lstlisting}[language=Python, basicstyle=\footnotesize, numbers=left, numberstyle=\tiny, breaklines=true, commentstyle=\itshape\color{magenta}, keywordstyle=\bfseries\color{blue}, identifierstyle=\color{cyan}, stringstyle=\color{red}, showstringspaces=false]

# ----------------------------
# AUTHOR: [REDACTED]
# crsID: [REDACTED]
# COLLEGE: [REDACTED]
# DATE: Feb 2021
# PURPOSE: Part III Project - Calculates and plots the gain against overvoltage of the SiPM by via the integral of the current. A linear regression is performed on the data.
# ----------------------------

import numpy as np
import matplotlib.pyplot as plt
import scipy.integrate as integrate
from scipy import optimize

counter = 0 # used to separate multiple data streams e.g. an IV graph with different resistance values

# ----------------------------
# SPECIFY FILE HERE:
# ----------------------------

f = open("FILE_PATH","r")

# ----------------------------
# Initialises global data arrays
# ----------------------------

# Ignore header lines
f.readline()
f.readline()

for line in f:
    line = line.strip()
    row = line.split()
    try: # test if the line contains data values
        floatTest = float(row[1])
    except ValueError:
        counter += 1

xVar = [[] for i in range(counter+1)]
yVar = [[] for i in range(counter+1)]
graphLabel = np.array([])

# ----------------------------
# Helper functions
# ----------------------------

def appendArray(data, count):
    global xVar, yVar
    xVar[count] = np.append(xVar[count],float(data[0]))
    yVar[count] = np.append(yVar[count],float(data[1]))
    return

def linFunc(x, m):
    return m*x

# ----------------------------
# Main function
# ----------------------------

# Reset back to beginning of the file
counter = 0
f.seek(0)
f.readline()

for line in f:
    # print(repr(line))
    line = line.strip()
    row = line.split()
    # variable[0] = np.append(variable[0],1.1)
    # print(variable)

    try: # test if the line contains data values
        floatTest = float(row[0])
    except ValueError:
        counter += 1
        graphLabel = np.append(graphLabel,row[2])
        continue # moves on to next line if the line does not contain data

    appendArray(row, counter - 1)

f.close()

# ----------------------------
# Calculate gain
# ----------------------------

ELECTRON_CHARGE = 1.6E-19

# SiPM 1 parameters
V_BREAK = 31.2
C_d = 34.6E-15
C_q = 12.2E-15
C_p = 23E-15

gain = np.zeros(len(xVar))
overvoltage = np.zeros(len(xVar))
theoretical = np.zeros(len(xVar))

for i in range(len(xVar)):
    gain[i] = np.trapz(yVar[i], xVar[i]) / ELECTRON_CHARGE
    overvoltage[i] = float(graphLabel[i][6:]) - V_BREAK

gain = gain - gain[0]

# ----------------------------
# Linear fit
# ----------------------------

params, covariance = optimize.curve_fit(linFunc, overvoltage, gain)
# print params

# ----------------------------
# Plot graphs
# ----------------------------

fig = plt.figure(figsize=(12, 7))
plt.plot(overvoltage, gain*1E-6, '+', markersize=10)
plt.plot(overvoltage, linFunc(overvoltage, params[0])*1E-6, 'k') # linear fit of the data

plt.xlabel('Overvoltage [V]')
plt.ylabel('Gain [x10$^{6}$]')
plt.legend(frameon=False)

plt.savefig("FILE_PATH")
plt.show()

\end{lstlisting}

\newpage

\section{SiPM 2 RESULTS}

\begin{figure}[h]
  \centering
  \includegraphics[width=0.7\linewidth]{Graphics/SiPM/SiPM2/"VariedCells"}
  {\caption*{Fig. D.1: Variation of the total number of microcells and the resultant voltage signals.}}
\end{figure}

\begin{figure}[h]
  \centering
  \includegraphics[width=0.7\linewidth]{Graphics/SiPM/SiPM2/"VariedFiredCells"}
  {\caption*{Fig. D.2: Variation of the number of fired microcells and the resultant voltage signals.}}
\end{figure}

\begin{figure}[h]
  \centering
  \includegraphics[width=0.7\linewidth]{Graphics/SiPM/SiPM2/"FiredCellsGain"}
  {\caption*{Fig. D.3: The relative gain from a single firing microcell up to the saturation of the SiPM ($N=400$).}}
\end{figure}

\begin{figure}[h]
  \centering
  \includegraphics[width=0.7\linewidth]{Graphics/SiPM/SiPM2/"VbiasGain"}
  {\caption*{Fig. D.4: $G=(2.89\times 10^5V^{-1})V_{ov}$}}
\end{figure}

\begin{figure}[h]
    \centering
    \begin{minipage}{0.5\linewidth}
        \centering
        \includegraphics[width=\linewidth]{Graphics/SiPM/SiPM2/"CdGain"}
        \caption*{(a)}
    \end{minipage}\hfill
    \begin{minipage}{0.5\textwidth}
        \centering
        \includegraphics[width=\linewidth]{Graphics/SiPM/SiPM2/"CqGain"}
        \caption*{(b)}
    \end{minipage}
    {\caption*{Fig. D.5: (a) $G=-(990fF^{-2})C_d^2+(2.27\times 10^5 fF^{-1})C_d-(7.06\times 10^6)$ and (b) $G=(2.91\times 10^5 fF^{-1})C_q-(5.90\times 10^6)$}}
\end{figure}

\begin{figure}[h]
    \centering
    \begin{minipage}{0.5\linewidth}
        \centering
        \includegraphics[width=\linewidth]{Graphics/SiPM/SiPM2/"TcycRecovery"}
        \caption*{(a)}
    \end{minipage}\hfill
    \begin{minipage}{0.5\textwidth}
        \centering
        \includegraphics[width=\linewidth]{Graphics/SiPM/SiPM2/"TcycRelPeak"}
        \caption*{(b)}
    \end{minipage}
    {\caption*{Fig. D.6: Recovery values of $\tau_{HP}=18.1ns$ and $\tau_{FP}=139ns$ are obtained.}}
\end{figure}

\begin{figure}[h]
    \centering
    \begin{minipage}{0.5\linewidth}
        \centering
        \includegraphics[width=\linewidth]{Graphics/SiPM/SiPM2/"VovTau"}
        \caption*{(a)}
    \end{minipage}\hfill
    \begin{minipage}{0.5\textwidth}
        \centering
        \includegraphics[width=\linewidth]{Graphics/SiPM/SiPM2/"RqTau"}
        \caption*{(b)}
    \end{minipage}
    \begin{minipage}{0.5\linewidth}
        \centering
        \includegraphics[width=\linewidth]{Graphics/SiPM/SiPM2/"CdTau"}
        \caption*{(c)}
    \end{minipage}\hfill
    \begin{minipage}{0.5\textwidth}
        \centering
        \includegraphics[width=\linewidth]{Graphics/SiPM/SiPM2/"CqTau"}
        \caption*{(d)}
    \end{minipage}
    {\caption*{\Fig. D.7: Effect of varying (a) $V_{ov}$, (b) $R_q$, (c) $C_d$ and (d) $C_q$ on the recovery time.}}
\end{figure}
