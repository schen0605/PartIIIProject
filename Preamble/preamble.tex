\usepackage[T1]{fontenc}
\usepackage[utf8x]{inputenc}
\usepackage{libertineMono-type1}
\usepackage[libertine]{newtxmath}
\renewcommand{\familydefault}{\sfdefault}

\usepackage{amssymb,amsmath, amsfonts}
\usepackage{booktabs}
\usepackage{soul}
\usepackage{color}
\usepackage{multirow}
% \usepackage{rotating}
\usepackage{graphicx}
% \usepackage{wasysym}
\usepackage{braket}
\usepackage{enumitem}
\usepackage[font=footnotesize]{caption}
\usepackage{listings}

% \usepackage{draftwatermark}
% \SetWatermarkText{Draft}
% \SetWatermarkScale{5}

\renewcommand*{\thefootnote}{\textbf{\arabic{footnote}}}

% \setlength{\voffset}{-0.75in}
% \setlength{\headsep}{5pt}
% \setlength{\textheight}{720pt}

\begin{document}

\begin{titlepage}
 \begin{center}
  \vspace*{2cm}

  \Huge
  \textbf{Simulation of Solid-State Sensors for a Spark Chamber Trigger System}

  \vspace{0.2cm}
  \LARGE
  Part III Project Report

  \vspace{2.5cm}

 \end{center}
 \begin{flushright}
  \Large
  \textbf{Supervisor: Dr. Steve Wotton}\\
  \textit{
   BGN:} 8234S \\
   \textit{May 2021}
\end{flushright}

\vfill
  \vspace{2cm}
  \Large
\noindent ABSTRACT:
    \normalsize

\noindent \textbf{Photomultiplier tubes (PMTs) are currently employed as part of a trigger system for a spark chamber held at the Cavendish Labratory. Advances in solid-state sensor technology have presented silicon photomultipliers (SiPMs) as a potential alternative to these traditional PMTs. In this project, we use circuit simulation software to simulate a SiPM using an equivalent circuit model. The results from the simulations are found to be consistent with theory.}

\noindent \textbf{In order to assess the feasibility of these devices in a trigger system, it is crucial to understand how a SiPM operates and its important features. Having a working model is very useful in this aspect and in future work, can be complementary to experimental measurements obtained from working with SiPMs.}
\\ \newline
\textit{Except where specific reference is made to the work of others, this work is original and has not been already submitted either wholly or in part to satisfy any degree requirement at this or any other university.}

% \vfill
% \noindent Supervisor Signature: \dotfill \ \ \ \ Student Signature: \dotfill
\end{titlepage}
