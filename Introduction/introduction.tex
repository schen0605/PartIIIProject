
\section{INTRODUCTION}

A spark chamber is a particle detector originally used as a research tool from the 1930s onwards and was a precursor to more sophisticated detectors such as bubble chambers in the 1960s. Nowadays, spark chambers are mostly found in museums or used as educational tools to promote particle physics and physics in general. \cite{HEPscatter}

\begin{figure}[h]
  \includegraphics[width=\linewidth]{Graphics/"SparkChamber1"}
  {\caption*{Fig. 1.1: Taken from \cite{HEPscatter}. The spark chamber built and used by the Cambridge HEP group.}}
\end{figure}

\noindent The High Energy Physics (HEP) group at Cambridge University has its own spark chamber for educational and promotional purposes (Fig. 1.1). By taking advantage of recent advances in solid-state photon sensor technology, it is believed that the photomultiplier tubes (PMTs) currently in use in the trigger system can be replaced in favour of these solid-state sensors. These devices have a number of advantages over traditional PMTs, most notably: a higher quantum efficiency\footnote{Quantum efficiency is a measure of a photodevice's effectiveness in converting incident photons into electrons. It indicates how much current is produced when photons of a given wavelength is absorbed.}, an insensitivity to magnetic fields and a more compact design. \cite{dinu2007}

Silicon photomultipliers (SiPMs) are becoming increasingly well established in various fields of physics. For example, the SiPM's capability to detect single photons and its excellent timing resolution (in the $200ps$ range) has made their use ideal in time-of-flight PET (TOF-PET). Their insensitivity to magnetic fields has also naturally led to their implementation in MR-based systems where PMTs were previously unsuitable. \cite{gundacker2020} In HEP, one of the earliest large scale adopters of SiPMs was in the T2K neutrino oscillation experiment. $60,000$ SiPMs were used - each made up of a customised $1.3\times 1.3 mm^2$ pixel with a $50\mu m$ microcell pitch. \cite{vacheret2011}

By utilising an analog circuit simulator, a SiPM is simulated using an equivalent circuit model. These simulations will help in understanding how a SiPM operates, which is crucial in assessing their suitability in a spark chamber trigger system. The features of a SiPM will need to be known and their impact carefully considered.
